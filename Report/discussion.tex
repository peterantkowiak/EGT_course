\section{Discussion and Conclusions}

\subsection{Main findings}

We found that people who oppose hosting more refugees tend to overestimate their real number whereas people who favor hosting more refugees tend to underestimate the real number of refugees.
However, our data do not support a correlation between knowledge and attitude.
While gender and age do not have a significant influence on attitude, answers differ significantly among places where the interviews were conducted.
Our main regression results give rise to discussion. Our data showed a significant correlation between estimates and attitude towards refugees. However, the precision of estimates did not correlate with attitude. According to \cite{Pettigrew1997}, \cite{Pettigrew1998} or \cite{Rydgren2004}, one would expect the estimates of people in favor of refugees to be more precise because they supposedly know more about them. In our study, the estimates of refugee-proponents were not more precise but significantly lower than the estimates of refugee-opponents. This result contrasts the theories of \cite{Pettigrew1997}, \cite{Pettigrew1998} and \cite{Rydgren2004}. We suppose that this difference may be due to several mechanisms. One explanation might be that most people know just so few about refugees that when asked to state their attitude, people can only rely on a gut feeling rather than real knowledge. Therefore, the subliminal perception of refugee presence might rather be the main factor shaping their attitude. According this line of reasoning, the interpretation of the estimates as a proxy of knowledge would have to be revised: The subtle perception of refugee presence might influence both people's attitude and how high they estimate the number of refugees.
In order to verify this interpretation and test whether above mentioned theories actually apply to the refugee issue, one might refine the interview scheme and include some questions to distinguish between gut feeling and real knowledge.

\subsection{Limitations}
While gender and age do not have a significant influence on attitude, answers differ significantly among places where the interviews were conducted. We therefore assume that the selection of interview sites is relevant for the outcome. Due to the limited time and number of researchers in our study, our results can not be fully generalized. For reproducing our study on a large scale one should aim to select the interviewees as representative as possible.\\
Besides the selection of interview sites, the scheme of interviews might constitute a certain limitation to our study. As the interviews were conducted orally, the responses might be influenced by interactions between interviewer and interviewees. For example, interviewees might attribute a pro-refugee opinion to the interviewer and therefore not be honest when having a different opinion themselves. This goes especially for interviewers speaking English. Interviewees might assume that they are talking to migrants and might therefore not want to express a negative attitude towards refugees. Several times it occurred that interviewees were in company of other people who would try to influence the answers by making comments. Although we explicitly asked the interviewees for their personal opinion, the results may have been biased in some cases. To increase objectivity and avoid bias by personal interactions, one could switch to anonymized printed questionnaires in future studies.\\

\subsection{Final conclusions, applications and further research}

In conclusion, we found that people who oppose hosting more refugees tend to overestimate the real numbers whereas people who favor hosting more refugees tend to underestimate the real numbers. However, our data do not support a correlation between precision of estimate and attitude. We cannot tell from our data if this means that subtle perception rather than knowledge influences people's attitude towards refugees. We therefore recommend further investigating this question with a more differentiated questionnaire that incorporates a well-defined proxy for the interviewees knowledge on refugees.\\
While gender and age did not significantly influence attitude in our study, answers differed significantly among places where the interviews were conducted. When reproducing this study, places for the interviews should be selected so that they represent an average of the population. Furthermore, we recommend using printed questionnaires to ensure objective and honest answers.
