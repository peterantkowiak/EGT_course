\section{Discussion and Conclusions}

\subsection{Main findings}

\subsubsection*{Spatial structure}


If the \textbf{Prisoner's Dilemma} is played in well-mixed populations, cooperation is not an evolutionarily stable strategy. However, previous research has shown that associative interactions, such as spatial structure can allow for the evolution of cooperation \citep{nowak1992,doebeli1998evolution,killingback1999variable}. In our simulations, we could reproduce this finding: In spatial PD games, cooperators form spatial clusters that reduce the exploitation by defectors.
However, cooperation could only be maintained at very low cost-benefit ratios ($r<0.1$). At higher r-values, defectors ``eat up'' the clusters from their borders so that the cooperators vanish from the population. Our finding matches well with the results of \cite{ohtsuki2006simple} who investigated different ratios of $benefit / cost$ found a threshold value: If $b/c$ is bigger than the average number of neighbors, cooperation can evolve. 


In the \textbf{Hawk-Dove Game}, the effect of spatial structure is more ambiguous. Due to the payoff structure of the HD game, it is most beneficial to use different strategies than neighboring cells. For this reason, even well-mixed populations maintain a ratio of $1-r$ cooperators. The same mechanism, however, inhibits the emergence of larger clusters. Instead, clusters in the shape of crosses or filaments are formed. Especially at higher r-values where the natural proportion of defectors is high, cooperators are much more prone to exploitation in the contact zones where they encounter adjacent defectors. As a consequence, the ratio of cooperators is lower than in well-mixed population for most (higher) r-values. Unlike in the well-mixed population, cooperation can vanish completely at high r-values. \\
At very low r-values, we found that cooperators profit from spatial structure. A possible explanation could be that even if the co-player defects, in the HD game the benefit of cooperation can still outweigh the cost. Furthermore, cooperators profit from cooperating neighbors, which are much more frequent at low r-values \citep{HauertandDoebeli2004}. 



\subsubsection*{Different neighborhood sizes}

Our simulations on the effect of varying neighborhood sizes again produced very different results for PD and HD games. 
In the \textbf{HD game}, both benefits and disadvantages through spatial structure were most pronounced in the small neighborhood ($4$ neighbors). The bigger the neighborhood, the curves leveled out and converged towards the linear $1-r$ relationship from the non-spatial HD game. We think that this effect is caused by the fact that in the HD game, spatial structure only works over very small distances because there are no larger clusters of the same strategy. When the co-player for the next round is drawn, in bigger neighborhoods the ratios of the different strategies are closer to the population average.
Besides the fact that spatial structure has a larger effect in small neighborhoods, we found that the extinction threshold for cooperators also varies with neighborhood size. This confirms the findings of \cite{HauertandDoebeli2004}.\\
Regarding the \textbf{PD game}, our results leave more room for interpretation.


\cite{ohtsuki2006simple}



\citep{wang2012spatial}


very small neighborhoods (4) are not sufficiently as capable to reduce exploitation from defectors as bigger ones
turning point / other processes

\subsubsection*{Mixed strategies}

compare the effect of spatial structure in the mixed-strategy HD game with that in the pure-strategy game






``What does cooperation mean in die Hawk-Dove game?''

Unlike in the PD game where a cooperator can only benefit if the co-player also cooperates, in the HD game the benefit of cooperation can still outweigh the cost.

\\
--> two variables influence whether a system is stable: neighborhood size and cost-benefit ratio

\\
(Ohtsuki et al 2006:
 ``natural selection favours cooperation, if the
 benefit of the altruistic act, b, divided by the cost, c, exceeds the
 average number of neighbours, k, which means b/c > k. In this
 case, cooperation can evolve as a consequence of ‘social viscosity’
 even in the absence of reputation effects or strategic complexity.'')
 

\subsection{Limitations}
While gender and age do not have a significant influence on attitude, answers differ significantly among places where the interviews were conducted. We therefore assume that the selection of interview sites is relevant for the outcome. Due to the limited time and number of researchers in our study, our results can not be fully generalized. For reproducing our study on a large scale one should aim to select the interviewees as representative as possible.\\
Besides the selection of interview sites, the scheme of interviews might constitute a certain limitation to our study. As the interviews were conducted orally, the responses might be influenced by interactions between interviewer and interviewees. For example, interviewees might attribute a pro-refugee opinion to the interviewer and therefore not be honest when having a different opinion themselves. This goes especially for interviewers speaking English. Interviewees might assume that they are talking to migrants and might therefore not want to express a negative attitude towards refugees. Several times it occurred that interviewees were in company of other people who would try to influence the answers by making comments. Although we explicitly asked the interviewees for their personal opinion, the results may have been biased in some cases. To increase objectivity and avoid bias by personal interactions, one could switch to anonymized printed questionnaires in future studies.\\

\subsection{Final conclusions, applications and further research}

\begin{itemize}
	\item{\textbf{spatial structure does not automatically make a system stable}}\\
	\item{\textbf{two variables influence whether a system is stable: neighborhood size and cost-benefit ratio}}
\end{itemize}


\begin{comment}
In conclusion, we found that people who oppose hosting more refugees tend to overestimate the real numbers whereas people who favor hosting more refugees tend to underestimate the real numbers. However, our data do not support a correlation between precision of estimate and attitude. We cannot tell from our data if this means that subtle perception rather than knowledge influences people's attitude towards refugees. We therefore recommend further investigating this question with a more differentiated questionnaire that incorporates a well-defined proxy for the interviewees knowledge on refugees.\\
While gender and age did not significantly influence attitude in our study, answers differed significantly among places where the interviews were conducted. When reproducing this study, places for the interviews should be selected so that they represent an average of the population. Furthermore, we recommend using printed questionnaires to ensure objective and honest answers.


It is widely assumed that spatial structure allows for the evolution of cooperation in PD games. However, this goes only for a small range of cost-benefit ratios. Cooperation is only evolutionarily stable for r-values $ \leq 0.09$, for bigger r-values it disappears from the population.\\

In spatial HD games: small neighborhoods = bigger profit when r is small, and small neighborhoods = bigger disadvantage when r is big.

In spatial PD games: 

Found anomality

r 0.03, nb4 -> stable at propC 0.35 after 2500 steps -> stable
r 0.065, nb4 -> cooperators die after 900 steps -> unstable
r 0.03, nb8 -> stable at propC 0.77 after 10000 steps -> stable
r 0.065, nb8 -> stable at propC 0.55 after 2500 steps -> stable
r 0.03, nb12 -> stable at propC 0.825 after 7500 steps -> stable
r 0.065, nb12 -> randomly oscillating around propC 0.33 after 10000 steps -> half-stable
r 0.03, nb24 -> stable at propC 0.85 after 4000 steps -> stable
r 0.065, nb 24 -> cooperators die after 7400 steps -> unstable


fixed r: at 0.03 and 0.065
varying neighborhood size
5000 steps, 5 repetitions

\end{comment}

