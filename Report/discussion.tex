\section{Discussion and Conclusions}


\subsection{Spatial structure}

If the \textbf{Prisoner's Dilemma} is played in well-mixed populations, cooperation is not an evolutionarily stable strategy. However, previous research has shown that associative interactions, such as spatial structure can allow for the evolution of cooperation \citep{nowak1992,doebeli1998evolution,killingback1999variable}. In our simulations, we could reproduce this finding: In spatial PD games, cooperators form spatial clusters that reduce the exploitation by defectors.
However, cooperation could only be maintained at very low cost-benefit ratios ($r<0.1$). At higher r-values, defectors ``eat up'' the clusters from their borders so that the cooperators vanish from the population. Our finding matches well with the results of \cite{ohtsuki2006simple} who investigated different ratios of $benefit / cost$ found a threshold value: If $b/c$ is bigger than the average number of neighbors, cooperation can evolve. 


In the \textbf{Hawk-Dove Game}, the effect of spatial structure is more ambiguous. Due to the payoff structure of the HD game, it is most beneficial to use different strategies than neighboring cells. For this reason, even well-mixed populations maintain a ratio of $1-r$ cooperators. The same mechanism, however, inhibits the emergence of larger clusters. Instead, clusters in the shape of crosses or filaments are formed. Especially at higher r-values where the natural proportion of defectors is high, cooperators are much more prone to exploitation in the contact zones where they encounter adjacent defectors. As a consequence, the ratio of cooperators is lower than in well-mixed population for most (higher) r-values. Unlike in the well-mixed population, cooperation can vanish completely at high r-values. \\
At very low r-values, we found that cooperators profit from spatial structure. A possible explanation could be that even if the co-player defects, in the HD game the benefit of cooperation can still outweigh the cost. Furthermore, cooperators profit from cooperating neighbors, which are much more frequent at low r-values \citep{HauertandDoebeli2004}. 



\subsection{Different neighborhood sizes}

Our simulations on the effect of varying neighborhood sizes again produced very different results for PD and HD games. 
In the \textbf{HD game}, both benefits and disadvantages through spatial structure were most pronounced in the small neighborhood ($4$ neighbors). The bigger the neighborhood, the curves leveled out and converged towards the linear $1-r$ relationship from the non-spatial HD game. We think that this effect is caused by the fact that in the HD game, spatial structure only works over very small distances because there are no larger clusters of the same strategy. When the co-player for the next round is drawn, in bigger neighborhoods the ratios of the different strategies are closer to the population average.
Besides the fact that spatial structure has a larger effect in small neighborhoods, we found that the extinction threshold for cooperators also varies with neighborhood size. This confirms the findings of \cite{HauertandDoebeli2004}.\\

\noindent Regarding the effects of different neighborhoods in the \textbf{PD game}, our results leave more room for interpretation. 
In our simulation, bigger neighborhoods resulted in higher frequencies of cooperation for r-values $\leq 0.05$. For higher r-values this effect partly reversed so that a neighborhood size of $8$ was optimal. At r-values $\geq 0.09$ cooperation died out for any neighborhood size (see figure \ref{fig: task2_multiplot}). We think that these somewhat ambiguous findings result from two different mechanisms that came into play at the same time:\\
Already in the first simulation, we were able to reproduce the common opinion that spatial structure favors cooperators in the PD game. The survival of cooperators in this case is facilitated by the formation of clusters: Patches in the center of a cluster receive the full benefit of cooperation at every time step. If the cost-benefit ratio is sufficiently low and the neighborhood sufficiently large, cooperating patches on the border of a cluster can profit from interaction with interior cells and compensate the loss that they suffer from adjacent defectors \citep{szabo2007evolutionary}.\\ 
The general mechanism of cluster formation works the better the larger the neighborhood of a single cell is: Larger clusters are formed and higher proportions of cooperators are maintained. Cooperator patches on the border of a cluster are less exposed to defectors the bigger a cluster is. In comparison, populations with very small neighborhoods ($4$ neighbors) can only form small clusters. Because the payoff of a small cooperator's cluster does not sufficiently exceed that of defectors, smaller clusters cannot reduce exploitation from defectors as efficiently as bigger clusters \citep{wang2012spatial}. We think that this mechanism explains the effect of neighborhood size found for r-values $\leq 0.05$.\\
Although populations with bigger neighborhood sizes are very stable for low r-values $\leq 0.05$ and maintain high rates of cooperators, the do not remain stable at higher r-values (figure \ref{fig: task2_radiusplot}). We think that this seemingly unexpected phenomenon is caused by the second mechanism: As our simulation lattice is only 50 x 50 patches large, the two biggest neighborhoods are prone to mean-field-type behavior. For low r-values, defection is not influential enough yet, but once the cooperating clusters start to struggle with high r-values, more defectors emerge and the cooperators die out due to the mean-field mechanism. We base our interpretation on the findings of \cite{wang2012spatial}, who recorded a very similar pattern in their PD simulations on regular lattices.



\subsection{Mixed strategies in the Hawk-Dove game}

In the mixed-strategy HD game, patches do not adopt one fixed strategy but rather inherit a certain probability to play either of the two strategies. In the pure-strategy non-spatial game, the evolutionarily stable frequency of cooperators is $1-r$. In the mixed-strategy non-spatial game, the evolutionarily stable probability to play cooperator in the next round is also $1-r$, which results in the same proportion of cooperators for both mixed and pure strategies. Once spatial structure is added to the model, the outcomes of mixed- and pure-strategy HD games differ though.
In general, spatial structure in the mixed-strategy HD game lowers the evolutionarily stable probability to cooperate. However, the detriment of spatial structure on cooperators is not as severe as in pure-strategy game. We think that this is because the mixed strategy adds another random element to the patches' strategy determination which counteracts the formation of spatial structures on the lattice.
In both mixed- and pure-strategy HD games, very high r-values lead to an extinction of cooperators (doves respectively) because there are so many defectors (hawks respectively) that almost any encounter leads to an escalating conflict \citep{HauertandDoebeli2004}.


\subsection{Final conclusions, applications and further research}

Based on our results, we conclude that spatial structure does not automatically enhance cooperation in the PD and HD games. In fact, this common belief only applies under certain conditions:\\
In the \textbf{Prisoner's Dilemma}, where cooperation is not evolutionarily stable in well-mixed population, spatial structure enables cooperation to persist at low cost-benefit ratios. Medium-large neighborhoods amplify this effect.\\
In the \textbf{Hawk-Dove game} however, spatial structure mostly reduces the frequency of cooperation which usually levels out at $1-r$ in well-mixed populations. Only in the rare case of pure-strategy games with small neighborhoods and very small r-values, cooperators can profit from spatial structure.\\
We find that neighborhood size and cost-benefit ratio (r) are the two most important variables determining whether a stable ratio of cooperators can persist in a population or not.\\
When applying Evolutionary Game Theory to natural populations, correctly measuring the payoffs is usually the most tricky part. If for example a high rate of cooperation is maintained in a population, it could either be the result of a spatial PD with small neighborhood size and low cost-benefit ratio, or just the cooperation rate naturally maintained in HD games - all depending on the payoff ranking. For this reason we aim to study more natural examples of PD and HD games and refine the available techniques for payoff measurement.
 




