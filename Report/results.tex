
\section{Results}

??? screenshots of spatial patterns ???


\subsection{Effect of spatial structure in PD and HD games}

In our first simulation experiment, we compared the effect of spatial structure on the persistence of cooperators in the PD and HD games. For the PD game we were able to reproduce the theoretical prediction that spatial structure enables cooperators to persist, even if cooperation is not an evolutionarily stable strategy in well-mixed populations. Our simulated spatial PD population with neighborhood size = 8 could maintain an average of $66.5\%$ cooperators ($\pm 1.12 \%$) at a cost-benefit ratio of r = 0.05. For higher cost-benefit ratios, however, cooperation was not evolutionarily stable at this neighborhood size and ceased within the 5000 time steps. If cooperation did was cost-free, the proportion of cooperators remained close to its initial value. See figure \ref{fig: task1_4plot} for a comparison of the frequency of cooperation in spatial and nonspatial PD games.



\begin{figure}[H]
	\centering 
	\includegraphics[width=9.5cm]{task1_4plot}
	\caption{Comparison of HD and PD game simulations, both with and without spatial structure.  \textbf{[ t = 5000, i = 10 ]} }\label{fig: task1_4plot}
\end{figure}






\subsection{Effect of neighbourhood size}

In the second simulation experiment we investigated the effect of 

\textbf{HD games}

\begin{figure}[H]
	\centering 
	\includegraphics[width=9.5cm]{task2_4plot}
	\caption{Effect of varying neighborhood size in the HD game.  \textbf{[ t = 5000, i = 10 ]} }\label{fig: task2_4plot}
\end{figure}



\textbf{PD games} 


\begin{figure}[H]
	\centering 
	\includegraphics[width=9.5cm]{task2_multiplot}
	\caption{Spatial PD game simulations with different neighborhood sizes.  \textbf{[ t = 5000, i = 10 ]} }\label{fig: task2_multiplot}
\end{figure}


It is widely assumed that spatial structure allows for the evolution of cooperation in PD games. However, this goes only for a small range of cost-benefit ratios. Cooperation is only evolutionarily stable for r-values $ \leq 0.09$, for bigger r-values it disappears from the population.\\
 
In spatial HD games: small neighborhoods = bigger profit when r is small, and small neighborhoods = bigger disadvantage when r is big.

In spatial PD games: 

Found anomality

r 0.03, nb4 -> stable at propC 0.35 after 2500 steps -> stable
r 0.065, nb4 -> cooperators die after 900 steps -> unstable
r 0.03, nb8 -> stable at propC 0.77 after 10000 steps -> stable
r 0.065, nb8 -> stable at propC 0.55 after 2500 steps -> stable
r 0.03, nb12 -> stable at propC 0.825 after 7500 steps -> stable
r 0.065, nb12 -> randomly oscillating around propC 0.33 after 10000 steps -> half-stable
r 0.03, nb24 -> stable at propC 0.85 after 4000 steps -> stable
r 0.065, nb 24 -> cooperators die after 7400 steps -> unstable

\begin{itemize}
\item{\textbf{spatial structure does not automatically make a system stable}}\\
\item{\textbf{two variables influence whether a system is stable: neighborhood size and cost-benefit ratio}}
\end{itemize}



fixed r: at 0.03 and 0.065
varying neighborhood size
5000 steps, 5 repetitions

\begin{figure}[H]
	\centering 
	\includegraphics[width=9.5cm]{task2_radiusplot}
	\caption{Spatial PD game simulations with fixed cost-benefit-ratio and different neighborhood sizes. Radius 1 is adequate to 4 neighbors, radius 1.4 = 8 neighbors, radius 2 = 12 neighbors and radius 2.8 = 24 neighbors.  \textbf{[ t = 10000, i = 10 ]} }\label{fig: task2_radiusplot}
\end{figure}


\subsection{Effect of mixed strategies}

In our third experiment, we compared the effect of spatial structure in the mixed-strategy HD game with that in the pure-strategy game.


\begin{figure}[H]
	\centering 
	\includegraphics[width=9.5cm]{task3_multiplot}
	\caption{Spatial HD game simulations with neighborhood size 8 and different strategies. The dotted black line depicts the frequency of cooperation in nonspatial games.  \textbf{[ t = 10000, i = 10 ]} }\label{fig: task3_multiplot}
\end{figure}





