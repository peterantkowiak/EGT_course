\section{Methods}

\subsection{Non-spatial PD and HD}

The first step for our model was to calculate the average payoff for cooperators and defectors in the non-spatial PD and HD. We used p for the probability of players being defectors (probability of cooperators is 1-p). Due to the payoff matrix of the HD game we used Pc = o.5*(1-p)*c+b-c and P_{d} = p*b. For the PD average pay-off for cooperators changes: P_{c} = (1-p)*b-c. These average Payoffs reduce or increase the fitness of the single players. In an iterated game fitter players reproduce and get a higher percentage of the population - in our case cooperators or defectors. For the next iteration (reproduction) of the game p is now calculated by comparing the fitness of defectors with the fitness of all players.
, the 
\subsection{Spatial structure and neighborhood size}
In our spatial games we have a 50 X 50 square lattice. Every square represents one player. The whole lattice is updated synchronically. We introduced different neighborhood-sizes from four to 24 neighbors in our model. Therefore we worked with five different sizes of radius (1, \sqrt[2]{2}, 2, 2*\sqrt[2]{2}, 3, caliber of players as unit) around the players for five neighborhood-sizes (4, 8, 12, 20, 24). Instead of using p, which was the probability to be a defector in the non-spatial game, we inserted a local probability pl in the average payoff terms. These were calculated with the neighbors' probabilities to be a cooperator or defector. Herewith fitness of the players is calculated again facing the neighborhood. As opposed to the non-spatial game the change of the strategy in the next round of the game is not calculated for the whole population, it is calculated for the single player (in our simulation patch). The players randomly choose neighbors for the competition. Then the transition probability (probability to change strategy) is calculated with p_{c} = z/\alpha. Z is the difference between the fitness of the competitor fc and the own fitness f. alpha is the maximum difference between the payoffs, which is alpha=T-P=b in the HD game and \alpha=T-S=b+c in the PD. This correction term ensures pc values between 0 and 1. Is z>0 the player changes the strategy with the probability p_{c}, which illustrates a reproduction of the fitter players.

\subsection{Effect of mixed strategies in the spatial and nonspatial Hawk-Dove game}
In our simulation for the HD with mixed strategies every player is characterized by the probability to play p to show dove-like behavior with an introduced mutation rate to do further exploration of the game. The initial heterogeneity of the players is randomly chosen from a normal distribution. The mean of that is the  equilibrium strategy of well-mixed populations p_{w} = (1-c/(2*b-v)), calculated from the cost-to-benefit ratio. The standard deviation is set to 0.02, the boundaries of the deviation are 0 and 1 to get a fitting value for the probability. The following procedure is the same as for the models with spatial structure, but with different mathematics to introduce the mutation. The average payoff P_{mix} = p_{w}*p_{n}*(b-(0.5*c))+ p_{w}*(1-p_{n})*(b-c)+(1-p_{w})*p_{n}*b is P with pn as the mean strategy of all interacting neighbors. More generally the term is P_{mix}  = p_{w}*p_{n}*R+ p_{w}*(1-p_{n})*S+(1-p_{w})*p_{n}*T+(1-p_{w})*(1-p_{n})*P. Now the  payoff differences between neighboring individuals is very small. The update rule for pure strategies has a very small probability of change, which makes the simulation very slow. For that reason a non-linear term is used for the change-probability: p_{cmix} = [1 + exp(-z/k)]^{-1}.

\subsection{Simulations with NetLogo, plots with R}
The simulations were programmed agent-based with NetLogo. The modeling of the non-spatial and the spatial HD and PD with different neighborhood-sizes were ran with the Behavior Space of the program. For having robust results we ran the simulation 10 times for 5000 time-steps with varying costs and benefit set to 1. The costs we calculated according to the replicator dynamics, the equilibrium frequency of cooperators in the HD with r=c/(2*b-c). Therefore c=(2*r/(1+r)) with a sequence for r from 0 to 1 with an 0.05 step. The population had the size  of 50 X 50 patches, that means 2500 players. The mixed strategy games ran for 10.000 time-steps to make sure that we have the equilibrium level. To compare the models we plotted the results in R with the frequency of cooperation on the y-axis and the cost-to-benefit ration on the x-axis. The results were compared with frequency of hawk-like behavior in well-mixed populations (f(wm)=1-r) visualized with a dotted line.

% für die weiteren plots, die du noch machst, kannst du dann evtl. noch kurz schreiben, was du da noch gemacht hast