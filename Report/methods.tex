\section{Methods}

\subsection{Modeling environment}

\subsection{Non-spatial Prisoner`s Dilemma and Hawk-Dove game}

The first step for our model was to calculating the average payoff for cooperators and defectors in the non-spatial Prisoner`s Dilemma and Hawk-Dove game. We worked with p for the probability of players being a  defectors (probability of cooperators is 1-p). Due to the payoff matrix of the Hawk-Dove game we used Pc = o.5*(1-p)*c+b-c and Pd = p*b. For the Prisoner`s Dilemma the average payoff for cooperator changes: Pc = (1-p)*b-c. These average Payoffs reduce or increase the fitness of the single players. In an iterated game fitter players reproduce and get an higher percentage of the population, in our case cooperators or defectors. For the next iteration (reproduction) of the game p is calculated new by comparing the fitness of defectors with the fitness of all players.
, the 
\subsection{Spatial structure and neighborhood size}
In our spatial games we have a 50 X 50 square lattice. Every square figures one player. The Whole lattice is updated synchronous. We introduced different neighborhood-sizes from four to 24 neighbors in our model. Therefore we worked with five different radiuses (1, sqrt2, 2, 2*sqrt2, 3,caaliber of players as unit) around the players for five neighborhood-sizes (4, 8, 12, 20, 24). Instead of using p, what was the probability to be a defector in the non-spatial game we insert a local probability pl in the average payoff therms which was calculated with the neighbors probabilities to be a cooperator or defector. With that the fitness of the players is calculated new facing the neighborhood. Else than in the non-spatial game the change of the strategy in the next round of the game is not calculated for the whole population, it is calculated for the single player (in our simulation patch). The players randomly choose neighbor for the competition. Than the transition probability (probability to change strategy) is calculated with pc = z/alpha. z is the difference between the fitness of the competitor fc and the own fitness f. alpha is the maximum difference between the payoffs, which is alpha=T-P=b in the Hawk-Dove game and alpha=T-S=b+c in the Prisoner`s Dilemma. normalisation. This corecction therm ensures pc values between 0 and 1. Is z>0 the player changes the strategy with the probability pc, what illustrates a reproduction of the fitter players.

\subsection{Effect of mixed strategies in the spatial and nonspatial Hawk-Dove game}
In our simulation for the Hawk-Dove game with mixed strategies every player is characterized by the probability to play p to show dove-like behavior with an introduced mutation rate to do further exploration. The initial heterogenity of the players is randomly chosen from an normal distribution. The mean of that is the  equilibrium strategy of well-mixed populations pw = (1-c/(2*b-v)), calculated from the cost-to-benefit ratio. The standard deviation is set to 0.02, the deviation borders??? are 0 and 1 to get worth ???for the  probability. The following procedure is the same like for the models with spatial structure, but with different mathematics to introduce the mutation. The average payoff Pmix = pw*pn*(b-(0.5*c))+ Pw*(1-pn)*(b-c)+(1-pw)*pn*b is P with pn as the mean strategy of all interacting neighbors. More general the term is Pmix = pw*pn*R+ Pw*(1-pn)*S+(1-pw)*pn*T+(1-pw)*(1-pn)*P. Now the  payoff differences between neighboring individuals is very small. The update rule for pure strategies has a very small probability of change, which makes the simulation very slowly. For that reason a non-linear therm is used for the change-probability: pcmix = [1 + exp(-z/k)]hoooch-1.

\subsection{Simulations with NetLogo, plots with R}
The simulations were programmed patch-based with NetLogo. blablabla. The modulations of the non-spatial and the spatial Hawk-Dove games and Prisoners Dilemma with different neighborhood-sizes were ran with the Behavior Space of the Program. For having robust results we ran the simulation 10 times for 5000 timesteps with varying costs (genauer!!!) and benefit set to 1.