\section{Introduction}

\subsection{The broad topic and what has been done before}

game theory
hd game
pd game 
pd hd and biology, processes in nature

The number of refugees asking for shelter in European Countries has strongly increased due to recent conflicts in the Near East and Africa.
From 336,000 refugees applying for asylum in the EU in 2012, the number of applications has increased to 437,000 in 2013. Because the applications were not equally distributed among the different EU countries, some were not affected at all whereas others had to make enormous efforts to accommodate the refugees. Asylum applications in the UK (+4\%) and France (+8\%) only slightly increased in 2013, while for example Italy (+61\%) and Germany (+64\%) experienced a much stronger increase \citep{BAMF2014}. In Germany as one of the main destination countries 127,000 refugees applied for asylum in 2013. 81,000 application were processed and only 20,000 refugees received a temporal or permanent residence permit \citep{BAMF2013}.
Now that many german municipalities struggle to find appropriate accommodation for the refugees, a public debate on whether or not to host more refugees has arisen. The opinions expressed differ largely among the population. While many people show honest empathy and try to help, there are many who do not want refugees living in proximity. Even utter xenophobia is pronounced occasionally. Media thereby often echo the most extreme positions.\\
In our perception the level of knowledge about refugees differs a lot among the population and the debate has turned very emotional so that it is often led by fear rather than facts. For this reason we want to investigate the public opinion and test whether people's attitudes are linked to their knowledge about refugees.

\textbf{Previous Research:}

There is a lot of research investigating the public opinion on immigrants, which can partly be transferred to the subgroup of refugees. Literature names several possible factors for negative attitudes towards asylum seekers. In general, the fear of crime and economic interests, such as the fear that migrants will take over all jobs have been identified as the most common reasons for negative attitudes towards refugees \citep{Otto2014}. Other authors focused on xenophobia and investigate whether it is related to gender \citep{Jolly2014} economics and education \citep{Francois2013}. Some studies have focused on the impact of media content on public opinion towards asylum seekers \citep{Boomgaarden2009, Perry1990, Brosius1995}. Furthermore, many authors have investigated how personal experience or contact influences people's attitudes \citep{Pettigrew1997}. \cite{Pettigrew1998} describes the underlying psychological processes of prejudice. He established the highly regarded "Intergroup Contact Theory". Its basic message is that a lack of knowledge about a social group leads to prejudice which in turn makes people avoid that group - a self enforcing process. Only direct contact with the social group can refute people's prejudices and bring about higher levels of knowledge and acceptance. \cite{Rydgren2004} stresses the coherence between lack of knowledge and xenophobia: "From an objective point of view, these kinds of beliefs are mostly non-rational or irrational because of their inaccurate correspondence with reality. Put another way, they are mostly false".

\subsection{Problem and Gap of Knowledge}

Previous research suggests that people who know less about the number of refugees are less tolerant and less likely to approve of more refugees being accepted. However this debate has not come to a final conclusion yet. Authors like \cite{Kahan2014} postulate that knowledge not necessarily correlates with what people think would be the right thing to do. We want to contribute to this discussion by investigating whether there people’s knowledge on refugees actually correlates with their support for Germany hosting more refugees.


\subsection{Our approach and specific questions}

\textbf{Hypothesis:}

In our study we test the following hypothesises: 

/itemize Spatial structure benefits cooperators especially in the Prisoner`s Dillema 
/itemize Neighborhood-size has ab influence to the effect of spatial structure
/itemize Mixed-strategies change the effect of spatial structure

For testing the Hypotheses we simulated different Hawk-Dove and Prisoners Dillema games with NetLogo 5.1.0 and performed them with varying variables. The results of the experiments were fitted and visualized in R to make them comparable. 

Our experiments examined following questions:

/itemize Which influence has spatial structure in the Hawk-Dove and Prisoners Dilema game?
/itemize Which effect have different neighborhood sizes to the games?
/itemize Which influence has a mixed strategy of the players in a spatial and a nonspatial Hawk-Dove game?

