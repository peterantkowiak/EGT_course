\section{Introduction}

\subsection{The broad topic and what has been done before}

game theory
hd game
pd game 
pd hd and biology, processes in nature
In 1944 the the economic scientist Neumann and Morgenstern developed the theory of games as an economic model. Much later, in the 70th the Evolutionary game theory originated. Instead of playing the game once like in the game theory the game in evolutionary game theory is played over and over again by biologically or socially conditionated players. in the most cases a player is randomly drawn from a large population and has a specific behavior \citep{weibull1997}
\subsection{Problem and Gap of Knowledge}
A lot of researchers of social, economic and biological science worked on the evolutionary game theory. It has become a powerful tool to investigate the emergence of cooperation in groups \citep{HauertandDoebeli2004}. For the Prisoner`s Dilemma it is widely accepted, that spatial structure supports cooperation \citep{margules2000}. To simulate biological processes there is an increasing discomfort with the Prisoner`s Dilemma as the only model to discuss cooperative behavior. The Hawk-Dove game is an interesting alternative for describing behavior pattern of field studies\citep{pressey1994}. The processes were discussed theoretically for several times \citep{margules2000, pressey1994}, but haven never been simulated, maybe due to a lack of computing power.  To find out the different performance of  the Prisoner`s Dilemma and  the  Hawk-Dove game, we compare the results of various simulations. For describing natural processes there has to be done further investigation of the Hawk-Dove game especially to simulate mutations of the behavior, here called Mixed-strategies. 
simulation something new

\subsection{Our approach and specific questions}
\textbf{Hypothesis:}

In our study we test the following hypotheses: 

\begin{itemize}
\item Spatial structure benefits cooperators especially in the Prisoner`s Dilemma 
\item Neighborhood-size has ab influence to the effect of spatial structure
\item Mixed-strategies change the effect of spatial structure
\end{itemize}

For testing the Hypotheses we simulated different Hawk-Dove and Prisoner`s Dilemma games with NetLogo 5.1.0 and performed them with varying variables. The results of the experiments were fitted and visualized in R to make them comparable. The focus of the Experiments Data was set on the frequency of cooperators with different cost to benefit ratios.

Our experiments examined following questions:
\begin{itemize}
\item Which influence has spatial structure in the Hawk-Dove and Prisoners Dilemma game?
\item Which effect have different neighborhood sizes to the games?
\item Which influence has a mixed strategy of the players in a spatial and a non-spatial Hawk-Dove game?
\end{itemize}
