\section{Introduction}

\subsection{The broad topic and what has been done before}

Game Theory

In 1944 the economic scientists Neumann and Morgenstern developed the theory of games as an economic model. Much later, in the 70s the evolutionary game theory originated. Instead of playing the game once, like in the game theory, the game in evolutionary game theory is played over and over again by biologically or socially conditioned players. In most cases a player is randomly drawn from a large population and has a specific behavior \citep{weibull1997} Our study compares these non-spatial games in well-mixed populations with spatial games and varying variables.

In general the Prisoner`s Dilemma (PD) and the Hawk-Dove game (HD) have the following notation for a two player game with the two strategies - cooperation or defection. R, T, S and P are the payoffs of the game.
\newline\\
\begin{tabular}{|c|c|c|}
		\hline  & C & D \\ 
		\hline C & R & S \\ 
		\hline C & T & P \\ 
		\hline 
\end{tabular} \\
(R: Reward; T: Temptation; S: Sucker's reward; P: Punishment; C: Cooperator; D: Defector)


\subsection{The Prisoner`s Dilemma:}
\newline\\
\begin{tabular}{|c|c|c|}
	\hline  & C & D \\ 
	\hline C & b-c & -c \\ 
	\hline D & b & o \\ 
	\hline 
\end{tabular} \\
(b: benefit, c: cost)
\newline\\
In a PD defection is the evolutionarily stable strategy. The defector gets the benefit b when he plays with a cooperator, who gets a punishment: The cost c. The relationship between payoffs is T > R > P > S. When cooperation is mutual, both have the benefit R=b-c, but pay a cost for that. Mutual defection results in Payoff P = 0 for both players. Because of the punishment of the cooperator, when the other player defects, it is best to defect, regardless of the co-players' decision \citep{HauertandDoebeli2004}. Though, in nature there are examples which prove, that not only defection can be evolutionarily stable. Alarm calls warn other animals from predators. It seems to be that cooperation only works when the cost-to-benefit ratio is not too high and spatial structure exists \citep{clutton1999}.

\subsection{The Hawk-Dove game:}
\newline\\
\begin{tabular}{|c|c|c|}
	\hline  & C & D \\ 
	\hline C & b-c/2 & b-c \\ 
	\hline D & b & o \\ 
	\hline 
\end{tabular} \\
\newline\\ 
Field and experimental studies had problems with the PD as the only model to discuss behavior of "players". Its difficult to estimate proper fitness payoffs. That caused various problems between theory and field - and experimental studies. Scientists needed another model for the payoffs of cooperative behavior \citep{milinski1997, nowak1992}. In the HD mutual cooperators are better off, because they share the cost of cooperation and receiving the whole benefit. Although cooperation gets rewarded, not punished by playing with a defector. The payoffs P and S have a reverse order in the HD which differs from the PD. P becomes the value b-c, so P and S have the reverse order. The new payoff matrix (T>R>S>P) leads to persistence of cooperative players beside defectors except for very high costs (2b>c>b>o) which would recover the PD. When b>c>0 the best action in the Hawk-Dove game depends on the co-player. In case that players always play the opposite strategy of their game partner, you have stable coexistence of cooperators and defectors in well-mixed populations - thus, an equillibrium of cooperators and defectors \citep{HauertandDoebeli2004}.

\subsection{Problem and Gap of Knowledge}
A lot of researchers of social, economic and biological science worked on the evolutionary game theory. It has become a powerful tool to investigate the emergence of cooperation in groups \citep{HauertandDoebeli2004}. For the PD it is widely accepted, that spatial structure supports cooperation \citep{margules2000}. To simulate biological processes there is an increasing discomfort with the PD being the only model to discuss cooperative behavior. The HD is an interesting alternative for describing behavior patterns of field studies\citep{milinski1987}. The processes were discussed theoretically for several times \citep{nowak1992, milinski1987}, but have never been simulated; maybe due to a lack of computing power. To find out the different performance of the PD and  the  HD, we compare the results of various simulations. Choosing the right model to describe natural processes is a big challenge for scientists. For describing them with the right model, there has to be done further investigation on the Hawk-Dove game, especially to simulate mutations of the behavior, here called Mixed-strategies. 

\subsection{Our approach and specific questions}
\textbf{Hypothesis:}

In our study we test the following hypotheses: 

\begin{itemize}
\item Spatial structure benefits cooperators especially in the PD
\item Neighborhood-size has ab influence to the effect of spatial structure
\item Mixed-strategies change the effect of spatial structure
\end{itemize}

For testing the Hypotheses we simulated different HD and PD games with NetLogo 5.1.0 and performed them with varying variables. The results of the experiments were fitted and visualized in R to make them comparable. The focus of the Experiments Data was set on the frequency of cooperators with different cost to benefit ratios.

Our experiments examined following questions:
\begin{itemize}
\item Which influence has spatial structure in the HD and PD game?
\item Which effect have different neighborhood sizes to the games?
\item Which influence has a mixed strategy of the players in a spatial and a non-spatial HD?
\end{itemize}
