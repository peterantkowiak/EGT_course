\begin{abstract}
The evolutionary game theory is an important tool to investigate biological evolution. The two most classic games to simulate the behavior of animals are the Hawk-Dove game and the Prisoner`s Dilemma. In the games the players contest over a sharable resource. The both models have different payoffs, for which reason it is important to know how a simulation reacts for diverse settings. That is required so simulate biological processes proper. We mainly explored the effect of spatial structure with different cost-to-benefit ratios. Furthermore we introduced variable neighborhood-sizes and a mutation-rate. The simulations were programmed with NetLogo and plotted in R. In the program the probabilities to be a cooperator or a defector are calculated according to the payoff matrix of the games. We compared the simulations using the frequency of the cooperators and the cost-to-benefit ratio.
\end{abstract}