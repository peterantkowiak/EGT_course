\begin{abstract}
	\noindent
view questions as proxies for attitude and knowledge and tested for coherence between the answers. Possible confounding factors such as age, gender and place were recorded and tested for significance.
We find that people who oppose hosting more refugees tend to overestimate the real numbers whereas people who favor hosting more refugees tend to underestimate the real numbers. However, our data do not support a correlation between knowledge (i.e. precision of estimate) and attitude. While gender and age do not have a significant influence, answers differ significantly among the places where the interviews were conducted. In conclusion, our study shows that attitudes towards refugees depend more on their perceived presence rather than real knowledge about them.

\end{abstract}