\begin{abstract}
Evolutionary Game Theory is an important tool to investigate biological evolution. The two most classic games to simulate the behavior of animals are the Hawk-Dove game and the Prisoner's Dilemma. In these games the players contest over a sharable resource. Because both models follow different payoff schemes, it is important to know how a simulation model reacts to different settings. Knowing the relevance of particular parameters can enhance simulations of biological processes. In this study, we explored how spatial structure and different cost-to-benefit ratios affect the frequency of cooperation in simulated populations. Furthermore we simulated variable neighborhood-sizes and a mutation-rate. The simulations were programmed with NetLogo and plotted in R. In the NetLogo models, the probabilities for different behaviors were calculated according to the payoff matrices of the games. We compared the simulations by recording the frequency of cooperators and the cost-to-benefit ratio.
Our main finding is that spatial structure does not automatically enhance cooperation in evolutionary games. In the Prisoner's Dilemma spatial structure allows for cooperation at low cost-benefit ratios whereas spatial structure mostly lowers the frequency of cooperation in the Hawk-Dove game. Neighborhood size and cost-benefit ratio are the two most important variables determining whether a stable ratio of cooperators can persist in a population or not.
\end{abstract}