\documentclass{article}

\usepackage{color}
\usepackage{xcolor}
\usepackage[]{hyperref}
\definecolor{darkblue}{rgb}{0,0,.5}

\hypersetup{
	colorlinks=true,
	breaklinks=true,
	linkcolor=darkblue,
	menucolor=darkblue,
	urlcolor=darkblue,
	citecolor=darkblue}

\usepackage{graphicx} % Grafik
\usepackage{multicol} % Tabellen
\setlength{\columnsep}{7mm}
\usepackage{multirow} % Tabellen
\usepackage{booktabs} % Tabellen
\usepackage{float}

\usepackage[margin=2cm]{geometry}
\newcommand{\HRule}{\rule{\linewidth}{0.5mm}}

%\usepackage{microtype}

%\usepackage{preprint}
%\usepackage{ConTeXt}
%\usepackage{fullpage}
\usepackage{hyperref}


% enhanced citation package 
\usepackage{natbib}
\bibpunct{(}{)}{;}{a}{}{,}  % to adjust punctuation in references

\title{Exploring the effects of spatial structure in EGT-games}
\author{Gregor Hys \and Peter Antkowiak}



% package fullpage



\begin{document}

\maketitle


\vfill

\begin{figure}[H]
	\centering
	\includegraphics[width=6cm] {demo} 	
	\caption{Protest for refugees}\label{fig: demo}
\end{figure}

\HRule \\[0.5cm]

\begin{abstract}
	\noindent
The number of refugees arriving in Europe has strongly increased over the past months. In Germany as one of the main destination countries, a public debate on whether or not to host more refugees has arisen. The opinions expressed largely differ among the population.
There are scientific concepts trying to explain attitudes towards social groups in general and xenophobia in particular. However, there is no scientific consensus on attitudes towards refugees.
In our study we test for a coherence between people's willingness to host more refugees and their knowledge about the actual refugee numbers in Germany. 
We used two interview questions as proxies for attitude and knowledge and tested for coherence between the answers. Possible confounding factors such as age, gender and place were recorded and tested for significance.
We find that people who oppose hosting more refugees tend to overestimate the real numbers whereas people who favor hosting more refugees tend to underestimate the real numbers. However, our data do not support a correlation between knowledge (i.e. precision of estimate) and attitude. While gender and age do not have a significant influence, answers differ significantly among the places where the interviews were conducted. In conclusion, our study shows that attitudes towards refugees depend more on their perceived presence rather than real knowledge about them.

\end{abstract}

%\tableofcontents

\newpage

\twocolumn
\sloppy

\section{Introduction}

\subsection{The broad topic and what has been done before}

game theory
hd game
pd game 
pd hd and biology, processes in nature

The number of refugees asking for shelter in European Countries has strongly increased due to recent conflicts in the Near East and Africa.
From 336,000 refugees applying for asylum in the EU in 2012, the number of applications has increased to 437,000 in 2013. Because the applications were not equally distributed among the different EU countries, some were not affected at all whereas others had to make enormous efforts to accommodate the refugees. Asylum applications in the UK (+4\%) and France (+8\%) only slightly increased in 2013, while for example Italy (+61\%) and Germany (+64\%) experienced a much stronger increase \citep{BAMF2014}. In Germany as one of the main destination countries 127,000 refugees applied for asylum in 2013. 81,000 application were processed and only 20,000 refugees received a temporal or permanent residence permit \citep{BAMF2013}.
Now that many german municipalities struggle to find appropriate accommodation for the refugees, a public debate on whether or not to host more refugees has arisen. The opinions expressed differ largely among the population. While many people show honest empathy and try to help, there are many who do not want refugees living in proximity. Even utter xenophobia is pronounced occasionally. Media thereby often echo the most extreme positions.\\
In our perception the level of knowledge about refugees differs a lot among the population and the debate has turned very emotional so that it is often led by fear rather than facts. For this reason we want to investigate the public opinion and test whether people's attitudes are linked to their knowledge about refugees.

\textbf{Previous Research:}

There is a lot of research investigating the public opinion on immigrants, which can partly be transferred to the subgroup of refugees. Literature names several possible factors for negative attitudes towards asylum seekers. In general, the fear of crime and economic interests, such as the fear that migrants will take over all jobs have been identified as the most common reasons for negative attitudes towards refugees \citep{Otto2014}. Other authors focused on xenophobia and investigate whether it is related to gender \citep{Jolly2014} economics and education \citep{Francois2013}. Some studies have focused on the impact of media content on public opinion towards asylum seekers \citep{Boomgaarden2009, Perry1990, Brosius1995}. Furthermore, many authors have investigated how personal experience or contact influences people's attitudes \citep{Pettigrew1997}. \cite{Pettigrew1998} describes the underlying psychological processes of prejudice. He established the highly regarded "Intergroup Contact Theory". Its basic message is that a lack of knowledge about a social group leads to prejudice which in turn makes people avoid that group - a self enforcing process. Only direct contact with the social group can refute people's prejudices and bring about higher levels of knowledge and acceptance. \cite{Rydgren2004} stresses the coherence between lack of knowledge and xenophobia: "From an objective point of view, these kinds of beliefs are mostly non-rational or irrational because of their inaccurate correspondence with reality. Put another way, they are mostly false".

\subsection{Problem and Gap of Knowledge}
A lot of researchers of social, economic or biological science worked on the evolutionary game theory. It has become a powerful tool to investigate the emergence of cooperation in groups \citep{HauertandDoebeli2004}. For the Prisoner`s Dilemma it is widely accepted, that spatial structure supports cooperation \citep{margules2000}. To simulate biological processes there is an increasing discomfort with the Prisoner`s Dilemma as the only model to discuss cooperative behavior. The Hawk-Dove game is an interesting alternative for describing behavior pattern of field studies\citep{pressey1994}. To find out the different performance of the simulations of the Prisoner`s Dilemma and  the  Hawk-Dove game we have to compare the results of the two games with different settings. In addition there has to be done further investigation of the Hawk-Dove game. 

\subsection{Our approach and specific questions}
\textbf{Hypothesis:}

In our study we test the following hypotheses: 

\begin{itemize}
\item Spatial structure benefits cooperators especially in the Prisoner`s Dilemma 
\item Neighborhood-size has ab influence to the effect of spatial structure
\item Mixed-strategies change the effect of spatial structure
\end{itemize}

For testing the Hypotheses we simulated different Hawk-Dove and Prisoner`s Dilemma games with NetLogo 5.1.0 and performed them with varying variables. The results of the experiments were fitted and visualized in R to make them comparable. The focus of the Experiments Data was set on the frequency of cooperators with different cost to benefit ratios.

Our experiments examined following questions:
\begin{itemize}
\item Which influence has spatial structure in the Hawk-Dove and Prisoners Dilemma game?
\item Which effect have different neighborhood sizes to the games?
\item Which influence has a mixed strategy of the players in a spatial and a non-spatial Hawk-Dove game?
\end{itemize}


\section{Methods}


\subsection{Non-spatial PD and HD}

The first step for our model was to calculate the average payoff for cooperators and defectors in the non-spatial PD and HD. We used p for the probability of players being defectors (probability of cooperators is $1-p$). Due to the payoff matrix of the HD game we used $P_{c}=0.5*(1-p)*c+b-c$ and $P_{d}=p*b$. For the PD average pay-off for cooperators changes: $P_{c}=(1-p)*b-c$. These average Payoffs reduce or increase the fitness of the single players. In an iterated game fitter players reproduce and get a higher percentage of the population - in our case cooperators or defectors. For the next iteration (reproduction) of the game p is now calculated by comparing the fitness of defectors with the fitness of all players.
 
\subsection{Spatial structure and neighborhood size}
In our spatial games we used a 50 x 50 square lattice. Every patch represents one player. The whole lattice was updated synchronically. We introduced different neighborhood-sizes from four to 24 neighbors in our model. Therefore we worked with four different sizes of radius ($1$, $\sqrt[2]{2}$, $2$, $2*\sqrt[2]{2}$, caliber of players as unit) around the players for five neighborhood-sizes (4, 8, 12, 24). Instead of using $p$, which was the probability to be a defector in the non-spatial game, we inserted a local probability $p_l$ in the average payoff terms. These were calculated with the neighbors' probabilities to be a cooperator or defector. Herewith fitness of the players is calculated again facing the neighborhood. As opposed to the non-spatial game the change of the strategy in the next round of the game was not calculated for the whole population, it was calculated for the single player (in our simulation patch). The players randomly chose neighbors for the competition. Then the transition probability (probability to change strategy) was calculated with $p_{c} = Z/\alpha$. $Z$ is the difference between the fitness of the competitor $F_c$ and the own fitness $F$. $\alpha$ is the maximum difference between the payoffs, and equals $\alpha=T-P=b$ in the HD game and $\alpha=T-S=b+c$ in the PD. This correction term ensures $p_{c}$ values between $0$ and $1$. If $Z>0$, the player changed the strategy with the probability $p_{c}$, which represents a reproduction of the fitter players.

\subsection{Effect of mixed strategies in the spatial and non-spatial Hawk-Dove game}
In our simulation for the HD with mixed strategies every player was characterized by the probability $p$ to show dove-like behavior which in turn was subject to a small mutation rate to allow for evolution in the game. The initial heterogeneity of the players was randomly chosen from a normal distribution. The mean of the normal distribution was the equilibrium strategy of well-mixed populations $p_{w} = (1-c/(2*b-v))$, calculated from the cost-to-benefit ratio. The standard deviation was set to $0.02$. The boundaries of the distribution were set to $0$ and $1$ to get a fitting value for the probability. The following procedure was the same as for the models with spatial structure, but with different mathematics to introduce the mutation. The average payoff $P_{mix} = p_{w}*p_{n}*(b-(0.5*c))+ p_{w}*(1-p_{n})*(b-c)+(1-p_{w})*p_{n}*b$ was $P$ with $p_n$ as the mean strategy of all interacting neighbors. More generally the term is $P_{mix}  = p_{w}*p_{n}*R+ p_{w}*(1-p_{n})*S+(1-p_{w})*p_{n}*T+(1-p_{w})*(1-p_{n})*P$, which means that the payoff differences between neighboring individuals are very small. The update rule for pure strategies had a very small probability of change, which made the simulation very slow. For that reason a non-linear term was introduced for the change-probability: $p_{cmix} = [1 + exp(-z/k)]^{-1}$.

\subsection{Simulations with NetLogo, plots with R}
The simulations were programmed agent-based with NetLogo \citep{Wilensky1999}. The modeling of the non-spatial and the spatial HD and PD with different neighborhood-sizes were ran with the Behavior Space of the program. For having robust results we ran the simulation 10 times for 5000 time-steps with varying costs and benefit set to 1. The costs we calculated according to the replicator dynamics, the equilibrium frequency of cooperators in the HD with $r=c/(2*b-c)$. Therefore $c=(2*r/(1+r))$ with a sequence for r from 0 to 1 with an 0.05 step. The population had the size  of 50 x 50 patches, that means 2500 players. The mixed strategy games ran for 10.000 time-steps to make sure that we the equilibrium level. To compare the models we plotted the results in R \citep{R} with the frequency of cooperation on the y-axis and the cost-to-benefit ration on the x-axis. 




\newpage

\section{Results}


\subsection{Linear Regression}
In our first linear regression model, we investigated whether there is a correlation between the estimated number of refugees and the tolerance statement. 
The regression shows a significant correlation between estimates and the attitude statement (figure \ref{fig: Regression}): The higher the estimated number of refugees, the more likely people oppose hosting more refugees.

\begin{figure}[H]
	\centering 
	\includegraphics[width=7cm]{Regression}
	\caption{R summary of the linear regression model}\label{fig: Regression}
\end{figure}

We produced a residual pattern plot and a Q-Q-Plot of the model to examine model quality. The residuals are nicely distributed in both Residual vs. fitted Plot and Q-Q-Plot (see figure \ref{fig: Q-Q-Plot}). Therefore we consider our model appropriate.
A summarizing box plot of our model including the regression line is given in figure \ref{fig: ModelPlot}.

In the second regression we correlated the logarithmized absolute difference between estimates and actual refugee number to the attitude records. Our data do not show a significant correlation between attitude and the difference to actual refugee numbers as a proxy for knowledge.



\begin{figure}[H]
	\centering 
	\includegraphics[width=7cm]{Q-Q-Plot}
	\caption{residual distribution and Q-Q-Plot of the fitted model}\label{fig: Q-Q-Plot}
\end{figure}


\begin{figure}[H]
	\centering 
	\includegraphics[width=7cm]{ModelPlot}
	\caption{Attitude towards refugees in dependence of estimated numbers. Blue = regression line}\label{fig: ModelPlot}
\end{figure}


\subsection{Confounding factors}

In order to check for possible confounding factors, we conducted an analysis of variance for gender, age and place of interview.

\textbf{Gender:} There was no influence of gender on attitude. Surprisingly, statements by male and female are completely equally distributed, even though we were 5 researchers interviewing 124 people. 

\textbf{Age:} According to preliminary data analysis, people favoring more refugees were younger than people who favored keeping the same number or less refugees. However, the ANOVA we conducted to test for an influence of age was not significant (p-value = 0.7898). We therefore do not consider age a relevant confounding factor.

\textbf{Place:} Unlike with the other variables, the place where interviews were conducted had a significant influence on the attitudes expressed. The ANOVA we returned a p-value of 0.0386.
In Vauban and on campus, people were more positive towards refugees than in the hiking area or at the technical faculty (see figure \ref{fig: Place}).

\begin{figure}
	\centering 
	\includegraphics[width=7cm]{Place}
	\caption{Attitudes towards refugees at different places}\label{fig: Place}
\end{figure}


\section{Discussion and Conclusions}

\subsection{Main findings}

\subsubsection*{Spatial structure}


If the \textbf{Prisoner's Dilemma} is played in well-mixed populations, cooperation is not an evolutionarily stable strategy. However, previous research has shown that associative interactions, such as spatial structure can allow for the evolution of cooperation \citep{nowak1992,doebeli1998evolution,killingback1999variable}. In our simulations, we could reproduce this finding: In spatial PD games, cooperators form spatial clusters that reduce the exploitation by defectors.
However, cooperation could only be maintained at very low cost-benefit ratios ($r<0.1$). At higher r-values, defectors ``eat up'' the clusters from their borders so that the cooperators vanish from the population. Our finding matches well with the results of \cite{ohtsuki2006simple} who investigated different ratios of $benefit / cost$ found a threshold value: If $b/c$ is bigger than the average number of neighbors, cooperation can evolve. 


In the \textbf{Hawk-Dove Game}, the effect of spatial structure is more ambiguous. Due to the payoff structure of the HD game, it is most beneficial to use different strategies than neighboring cells. For this reason, even well-mixed populations maintain a ratio of $1-r$ cooperators. The same mechanism, however, inhibits the emergence of larger clusters. Instead, clusters in the shape of crosses or filaments are formed. Especially at higher r-values where the natural proportion of defectors is high, cooperators are much more prone to exploitation in the contact zones where they encounter adjacent defectors. As a consequence, the ratio of cooperators is lower than in well-mixed population for most (higher) r-values. Unlike in the well-mixed population, cooperation can vanish completely at high r-values. \\
At very low r-values, we found that cooperators profit from spatial structure. A possible explanation could be that even if the co-player defects, in the HD game the benefit of cooperation can still outweigh the cost. Furthermore, cooperators profit from cooperating neighbors, which are much more frequent at low r-values \citep{HauertandDoebeli2004}. 



\subsubsection*{Different neighborhood sizes}

Our simulations on the effect of varying neighborhood sizes again produced very different results for PD and HD games. 
In the \textbf{HD game}, both benefits and disadvantages through spatial structure were most pronounced in the small neighborhood ($4$ neighbors). The bigger the neighborhood, the curves leveled out and converged towards the linear $1-r$ relationship from the non-spatial HD game. We think that this effect is caused by the fact that in the HD game, spatial structure only works over very small distances because there are no larger clusters of the same strategy. When the co-player for the next round is drawn, in bigger neighborhoods the ratios of the different strategies are closer to the population average.
Besides the fact that spatial structure has a larger effect in small neighborhoods, we found that the extinction threshold for cooperators also varies with neighborhood size. This confirms the findings of \cite{HauertandDoebeli2004}.\\
Regarding the \textbf{PD game}, our results leave more room for interpretation.


\cite{ohtsuki2006simple}



\citep{wang2012spatial}


very small neighborhoods (4) are not sufficiently as capable to reduce exploitation from defectors as bigger ones
turning point / other processes

\subsubsection*{Mixed strategies}

compare the effect of spatial structure in the mixed-strategy HD game with that in the pure-strategy game






``What does cooperation mean in die Hawk-Dove game?''

Unlike in the PD game where a cooperator can only benefit if the co-player also cooperates, in the HD game the benefit of cooperation can still outweigh the cost.

\\
--> two variables influence whether a system is stable: neighborhood size and cost-benefit ratio

\\
(Ohtsuki et al 2006:
 ``natural selection favours cooperation, if the
 benefit of the altruistic act, b, divided by the cost, c, exceeds the
 average number of neighbours, k, which means b/c > k. In this
 case, cooperation can evolve as a consequence of ‘social viscosity’
 even in the absence of reputation effects or strategic complexity.'')
 

\subsection{Limitations}
While gender and age do not have a significant influence on attitude, answers differ significantly among places where the interviews were conducted. We therefore assume that the selection of interview sites is relevant for the outcome. Due to the limited time and number of researchers in our study, our results can not be fully generalized. For reproducing our study on a large scale one should aim to select the interviewees as representative as possible.\\
Besides the selection of interview sites, the scheme of interviews might constitute a certain limitation to our study. As the interviews were conducted orally, the responses might be influenced by interactions between interviewer and interviewees. For example, interviewees might attribute a pro-refugee opinion to the interviewer and therefore not be honest when having a different opinion themselves. This goes especially for interviewers speaking English. Interviewees might assume that they are talking to migrants and might therefore not want to express a negative attitude towards refugees. Several times it occurred that interviewees were in company of other people who would try to influence the answers by making comments. Although we explicitly asked the interviewees for their personal opinion, the results may have been biased in some cases. To increase objectivity and avoid bias by personal interactions, one could switch to anonymized printed questionnaires in future studies.\\

\subsection{Final conclusions, applications and further research}

\begin{itemize}
	\item{\textbf{spatial structure does not automatically make a system stable}}\\
	\item{\textbf{two variables influence whether a system is stable: neighborhood size and cost-benefit ratio}}
\end{itemize}


\begin{comment}
In conclusion, we found that people who oppose hosting more refugees tend to overestimate the real numbers whereas people who favor hosting more refugees tend to underestimate the real numbers. However, our data do not support a correlation between precision of estimate and attitude. We cannot tell from our data if this means that subtle perception rather than knowledge influences people's attitude towards refugees. We therefore recommend further investigating this question with a more differentiated questionnaire that incorporates a well-defined proxy for the interviewees knowledge on refugees.\\
While gender and age did not significantly influence attitude in our study, answers differed significantly among places where the interviews were conducted. When reproducing this study, places for the interviews should be selected so that they represent an average of the population. Furthermore, we recommend using printed questionnaires to ensure objective and honest answers.


It is widely assumed that spatial structure allows for the evolution of cooperation in PD games. However, this goes only for a small range of cost-benefit ratios. Cooperation is only evolutionarily stable for r-values $ \leq 0.09$, for bigger r-values it disappears from the population.\\

In spatial HD games: small neighborhoods = bigger profit when r is small, and small neighborhoods = bigger disadvantage when r is big.

In spatial PD games: 

Found anomality

r 0.03, nb4 -> stable at propC 0.35 after 2500 steps -> stable
r 0.065, nb4 -> cooperators die after 900 steps -> unstable
r 0.03, nb8 -> stable at propC 0.77 after 10000 steps -> stable
r 0.065, nb8 -> stable at propC 0.55 after 2500 steps -> stable
r 0.03, nb12 -> stable at propC 0.825 after 7500 steps -> stable
r 0.065, nb12 -> randomly oscillating around propC 0.33 after 10000 steps -> half-stable
r 0.03, nb24 -> stable at propC 0.85 after 4000 steps -> stable
r 0.065, nb 24 -> cooperators die after 7400 steps -> unstable


fixed r: at 0.03 and 0.065
varying neighborhood size
5000 steps, 5 repetitions

\end{comment}





\section*{Acknowledgements}
First of all we want to thank all participants of the interviews who have made the whole study possible. We also want to thank the R core team for their great statistics software. Sincere thanks are due to our teachers Dr. Florian Hartig and Paul Bauche for their competent support and advice. Special thanks go to Lisa Rubin for her helpful review of the manuscript.



\onecolumn

\newpage
\newpage

\bibliographystyle{chicago}
\bibliography{EGT_refs}


\end{document}