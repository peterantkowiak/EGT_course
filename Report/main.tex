\documentclass{article}

\usepackage{color}
\usepackage{xcolor}
\usepackage[]{hyperref}
\definecolor{darkblue}{rgb}{0,0,.5}

\hypersetup{
	colorlinks=true,
	breaklinks=true,
	linkcolor=darkblue,
	menucolor=darkblue,
	urlcolor=darkblue,
	citecolor=darkblue}

\usepackage{graphicx} % Grafik
\usepackage{multicol} % Tabellen
\setlength{\columnsep}{7mm}
\setlength{\parindent}{15pt}
\usepackage{multirow} % Tabellen
\usepackage{booktabs} % Tabellen
\usepackage{float}

\usepackage[top=1.2cm, bottom=1.5cm, left=1cm, right=1cm]{geometry}
\newcommand{\HRule}{\rule{\linewidth}{0.5mm}}

%\usepackage{microtype}

%\usepackage{preprint}
%\usepackage{ConTeXt}
%\usepackage{fullpage}
\usepackage{hyperref}


% enhanced citation package 
\usepackage{natbib}
\bibpunct{(}{)}{;}{a}{}{,}  % to adjust punctuation in references

\title{Exploring the effects of spatial structure in EGT-games}
\author{Gregor Hys \and Peter Antkowiak}



% package fullpage



\begin{document}

\maketitle


\vfill

\begin{figure}[H]
	\centering
	\includegraphics[width=6cm] {demo} 	
	\caption{Protest for refugees}\label{fig: demo}
\end{figure}

\HRule \\[0.5cm]

\begin{abstract}
	\noindent
The number of refugees arriving in Europe has strongly increased over the past months. In Germany as one of the main destination countries, a public debate on whether or not to host more refugees has arisen. The opinions expressed largely differ among the population.
There are scientific concepts trying to explain attitudes towards social groups in general and xenophobia in particular. However, there is no scientific consensus on attitudes towards refugees.
In our study we test for a coherence between people's willingness to host more refugees and their knowledge about the actual refugee numbers in Germany. 
We used two interview questions as proxies for attitude and knowledge and tested for coherence between the answers. Possible confounding factors such as age, gender and place were recorded and tested for significance.
We find that people who oppose hosting more refugees tend to overestimate the real numbers whereas people who favor hosting more refugees tend to underestimate the real numbers. However, our data do not support a correlation between knowledge (i.e. precision of estimate) and attitude. While gender and age do not have a significant influence, answers differ significantly among the places where the interviews were conducted. In conclusion, our study shows that attitudes towards refugees depend more on their perceived presence rather than real knowledge about them.

\end{abstract}

%\tableofcontents

\newpage

\twocolumn
\sloppy


\section{Introduction}

\subsection{The broad topic and what has been done before}

game theory
hd game
pd game 
pd hd and biology, processes in nature

The number of refugees asking for shelter in European Countries has strongly increased due to recent conflicts in the Near East and Africa.
From 336,000 refugees applying for asylum in the EU in 2012, the number of applications has increased to 437,000 in 2013. Because the applications were not equally distributed among the different EU countries, some were not affected at all whereas others had to make enormous efforts to accommodate the refugees. Asylum applications in the UK (+4\%) and France (+8\%) only slightly increased in 2013, while for example Italy (+61\%) and Germany (+64\%) experienced a much stronger increase \citep{BAMF2014}. In Germany as one of the main destination countries 127,000 refugees applied for asylum in 2013. 81,000 application were processed and only 20,000 refugees received a temporal or permanent residence permit \citep{BAMF2013}.
Now that many german municipalities struggle to find appropriate accommodation for the refugees, a public debate on whether or not to host more refugees has arisen. The opinions expressed differ largely among the population. While many people show honest empathy and try to help, there are many who do not want refugees living in proximity. Even utter xenophobia is pronounced occasionally. Media thereby often echo the most extreme positions.\\
In our perception the level of knowledge about refugees differs a lot among the population and the debate has turned very emotional so that it is often led by fear rather than facts. For this reason we want to investigate the public opinion and test whether people's attitudes are linked to their knowledge about refugees.

\textbf{Previous Research:}

There is a lot of research investigating the public opinion on immigrants, which can partly be transferred to the subgroup of refugees. Literature names several possible factors for negative attitudes towards asylum seekers. In general, the fear of crime and economic interests, such as the fear that migrants will take over all jobs have been identified as the most common reasons for negative attitudes towards refugees \citep{Otto2014}. Other authors focused on xenophobia and investigate whether it is related to gender \citep{Jolly2014} economics and education \citep{Francois2013}. Some studies have focused on the impact of media content on public opinion towards asylum seekers \citep{Boomgaarden2009, Perry1990, Brosius1995}. Furthermore, many authors have investigated how personal experience or contact influences people's attitudes \citep{Pettigrew1997}. \cite{Pettigrew1998} describes the underlying psychological processes of prejudice. He established the highly regarded "Intergroup Contact Theory". Its basic message is that a lack of knowledge about a social group leads to prejudice which in turn makes people avoid that group - a self enforcing process. Only direct contact with the social group can refute people's prejudices and bring about higher levels of knowledge and acceptance. \cite{Rydgren2004} stresses the coherence between lack of knowledge and xenophobia: "From an objective point of view, these kinds of beliefs are mostly non-rational or irrational because of their inaccurate correspondence with reality. Put another way, they are mostly false".

\subsection{Problem and Gap of Knowledge}
A lot of researchers of social, economic or biological science worked on the evolutionary game theory. It has become a powerful tool to investigate the emergence of cooperation in groups \citep{HauertandDoebeli2004}. For the Prisoner`s Dilemma it is widely accepted, that spatial structure supports cooperation \citep{margules2000}. To simulate biological processes there is an increasing discomfort with the Prisoner`s Dilemma as the only model to discuss cooperative behavior. The Hawk-Dove game is an interesting alternative for describing behavior pattern of field studies\citep{pressey1994}. To find out the different performance of the simulations of the Prisoner`s Dilemma and  the  Hawk-Dove game we have to compare the results of the two games with different settings. In addition there has to be done further investigation of the Hawk-Dove game. 

\subsection{Our approach and specific questions}
\textbf{Hypothesis:}

In our study we test the following hypotheses: 

\begin{itemize}
\item Spatial structure benefits cooperators especially in the Prisoner`s Dilemma 
\item Neighborhood-size has ab influence to the effect of spatial structure
\item Mixed-strategies change the effect of spatial structure
\end{itemize}

For testing the Hypotheses we simulated different Hawk-Dove and Prisoner`s Dilemma games with NetLogo 5.1.0 and performed them with varying variables. The results of the experiments were fitted and visualized in R to make them comparable. The focus of the Experiments Data was set on the frequency of cooperators with different cost to benefit ratios.

Our experiments examined following questions:
\begin{itemize}
\item Which influence has spatial structure in the Hawk-Dove and Prisoners Dilemma game?
\item Which effect have different neighborhood sizes to the games?
\item Which influence has a mixed strategy of the players in a spatial and a non-spatial Hawk-Dove game?
\end{itemize}


\section{Methods}

\subsection{Modeling environment}

\subsection{Non-spatial Prisoner`s Dilemma and Hawk-Dove game}

The first step for our model was to calculating the average payoff for cooperators and defectors in the non-spatial Prisoner`s Dilemma and Hawk-Dove game. We worked with p for the probability of players being a  defectors (probability of cooperators is 1-p). Due to the payoff matrix of the Hawk-Dove game we used Pc = o.5*(1-p)*c+b-c and Pd = p*b. For the Prisoner`s Dilemma the average payoff for cooperator changes: Pc = (1-p)*b-c. These average Payoffs reduce or increase the fitness of the single players. In an iterated game fitter players reproduce and get an higher percentage of the population, in our case cooperators or defectors. For the next iteration (reproduction) of the game p is calculated new by comparing the fitness of defectors with the fitness of all players.

\subsection{Spatial structure and neighborhood size}
In our spatial games we have a 50 X 50 square lattice. Every square figures one player. The Whole lattice is updated synchronous. We introduced different neighborhood-sizes from four to 24 neighbors in our model. Therefore we worked with five different radiuses (1, sqrt2, 2, 2*sqrt2, 3,caaliber of players as unit) around the players for five neighborhood-sizes (4, 8, 12, 20, 24). Instead of using p, what was the probability to be a defector in the non-spatial game we insert a local probability pl in the average payoff therms which was calculated with the neighbors probabilities to be a cooperator or defector. With that the fitness of the players is calculated new facing the neighborhood.



\textbf{Preparation:} For data analysis we used the R statistics software package \citep{RCoreTeam2014}. For a preliminary evaluation of the data, we plotted some parameters and created a histogram of the estimate \ref{fig: Histogram1}. Because of the highly skewed distribution of estimates we logarithmized the estimates to the base of 20,000 (i.e. the real number of refugees admitted) \ref{fig: Histogram2}.
The answers to Question 1 were numerified for analysis: \\
\\
\indent\indent less refugees = -1\\
\indent\indent same number = 0\\
\indent\indent more refugees = 1\\
\\
\textbf{Regression:} After preparing the data we applied two linear regression models to the data. First, we checked for a correlation between the attitude records and the logarithmized absolute estimates. We then calculated the difference between the estimates and the real number of refugees and checked for a correlation between attitude records and the logarithmized absolute difference.


\noindent\textbf{Confoundig factors:} In order to check whether one of the possible confounding factors had an influence on the attitude towards refugees, we conducted an analysis of variance for each of them.


%\onecolumn

\begin{figure}[H]
	\centering 
	\includegraphics[width=7cm]{Histogram}
	\caption{Histograms of the estimates}\label{fig: Histogram1}
\end{figure}
% schöner machen

\begin{figure}[H]
	\centering 
	\includegraphics[width=7cm]{LogHistogram}
	\caption{Histograms of the logarithmized estimates}\label{fig: Histogram2}
\end{figure}

%\twocolumn



\subsection{Effect of mixed strategies}






\section{Results}

??? screenshots of spatial patterns ???


\subsection{Effect of spatial structure in PD and HD games}

In our first simulation experiment, we compared the effect of spatial structure on the persistence of cooperators in the PD and HD games. For the PD game we were able to reproduce the theoretical prediction that spatial structure enables cooperators to persist, even if cooperation is not an evolutionarily stable strategy in well-mixed populations. Our simulated spatial PD population with neighborhood size = 8 could maintain an average of $66.5\%$ cooperators ($\pm 1.12 \%$) at a cost-benefit ratio of r = 0.05. For higher cost-benefit ratios, however, cooperation was not evolutionarily stable at this neighborhood size and ceased within the 5000 time steps. If cooperation did was cost-free, the proportion of cooperators remained close to its initial value. See figure \ref{fig: task1_4plot} for a comparison of the frequency of cooperation in spatial and nonspatial PD games.



\begin{figure}[H]
	\centering 
	\includegraphics[width=9.5cm]{task1_4plot}
	\caption{Comparison of HD and PD game simulations, both with and without spatial structure.  \textbf{[ t = 5000, i = 10 ]} }\label{fig: task1_4plot}
\end{figure}






\subsection{Effect of neighbourhood size}

In the second simulation experiment we investigated the effect of 

\textbf{HD games}

\begin{figure}[H]
	\centering 
	\includegraphics[width=9.5cm]{task2_4plot}
	\caption{Effect of varying neighborhood size in the HD game.  \textbf{[ t = 5000, i = 10 ]} }\label{fig: task2_4plot}
\end{figure}



\textbf{PD games} 


\begin{figure}[H]
	\centering 
	\includegraphics[width=9.5cm]{task2_multiplot}
	\caption{Spatial PD game simulations with different neighborhood sizes.  \textbf{[ t = 5000, i = 10 ]} }\label{fig: task2_multiplot}
\end{figure}


It is widely assumed that spatial structure allows for the evolution of cooperation in PD games. However, this goes only for a small range of cost-benefit ratios. Cooperation is only evolutionarily stable for r-values $ \leq 0.09$, for bigger r-values it disappears from the population.\\
 
In spatial HD games: small neighborhoods = bigger profit when r is small, and small neighborhoods = bigger disadvantage when r is big.

In spatial PD games: 

Found anomality

r 0.03, nb4 -> stable at propC 0.35 after 2500 steps -> stable
r 0.065, nb4 -> cooperators die after 900 steps -> unstable
r 0.03, nb8 -> stable at propC 0.77 after 10000 steps -> stable
r 0.065, nb8 -> stable at propC 0.55 after 2500 steps -> stable
r 0.03, nb12 -> stable at propC 0.825 after 7500 steps -> stable
r 0.065, nb12 -> randomly oscillating around propC 0.33 after 10000 steps -> half-stable
r 0.03, nb24 -> stable at propC 0.85 after 4000 steps -> stable
r 0.065, nb 24 -> cooperators die after 7400 steps -> unstable

\begin{itemize}
\item{\textbf{spatial structure does not automatically make a system stable}}\\
\item{\textbf{two variables influence whether a system is stable: neighborhood size and cost-benefit ratio}}
\end{itemize}



fixed r: at 0.03 and 0.065
varying neighborhood size
5000 steps, 5 repetitions

\begin{figure}[H]
	\centering 
	\includegraphics[width=9.5cm]{task2_radiusplot}
	\caption{Spatial PD game simulations with fixed cost-benefit-ratio and different neighborhood sizes. Radius 1 is adequate to 4 neighbors, radius 1.4 = 8 neighbors, radius 2 = 12 neighbors and radius 2.8 = 24 neighbors.  \textbf{[ t = 10000, i = 10 ]} }\label{fig: task2_radiusplot}
\end{figure}


\subsection{Effect of mixed strategies}

In our third experiment, we compared the effect of spatial structure in the mixed-strategy HD game with that in the pure-strategy game.


\begin{figure}[H]
	\centering 
	\includegraphics[width=9.5cm]{task3_multiplot}
	\caption{Spatial HD game simulations with neighborhood size 8 and different strategies. The dotted black line depicts the frequency of cooperation in nonspatial games.  \textbf{[ t = 10000, i = 10 ]} }\label{fig: task3_multiplot}
\end{figure}







\section{Discussion and Conclusions}

\subsection{Main findings}

\subsubsection*{Spatial structure}

compare the effect of spatial structure on the persistence of cooperators in PD and HD !!! see paper!!!

\subsubsection*{Different neighborhood sizes}


spatial HD: cooperation extinguishing threshold $1 / N > 1 - r$
spatial PD: Ohtsuki et al 2006

very small neighborhoods (4) are not sufficiently as capable to reduce exploitation from defectors as bigger ones
turning point / other processes

\subsubsection*{Mixed strategies}

compare the effect of spatial structure in the mixed-strategy HD game with that in the pure-strategy game





``What does cooperation mean in die Hawk-Dove game?''

\\
--> two variables influence whether a system is stable: neighborhood size and cost-benefit ratio

\\
(Ohtsuki et al 2006:
 ``natural selection favours cooperation, if the
 benefit of the altruistic act, b, divided by the cost, c, exceeds the
 average number of neighbours, k, which means b/c > k. In this
 case, cooperation can evolve as a consequence of ‘social viscosity’
 even in the absence of reputation effects or strategic complexity.'')
 

\subsection{Limitations}
While gender and age do not have a significant influence on attitude, answers differ significantly among places where the interviews were conducted. We therefore assume that the selection of interview sites is relevant for the outcome. Due to the limited time and number of researchers in our study, our results can not be fully generalized. For reproducing our study on a large scale one should aim to select the interviewees as representative as possible.\\
Besides the selection of interview sites, the scheme of interviews might constitute a certain limitation to our study. As the interviews were conducted orally, the responses might be influenced by interactions between interviewer and interviewees. For example, interviewees might attribute a pro-refugee opinion to the interviewer and therefore not be honest when having a different opinion themselves. This goes especially for interviewers speaking English. Interviewees might assume that they are talking to migrants and might therefore not want to express a negative attitude towards refugees. Several times it occurred that interviewees were in company of other people who would try to influence the answers by making comments. Although we explicitly asked the interviewees for their personal opinion, the results may have been biased in some cases. To increase objectivity and avoid bias by personal interactions, one could switch to anonymized printed questionnaires in future studies.\\

\subsection{Final conclusions, applications and further research}

\begin{itemize}
	\item{\textbf{spatial structure does not automatically make a system stable}}\\
	\item{\textbf{two variables influence whether a system is stable: neighborhood size and cost-benefit ratio}}
\end{itemize}


\begin{comment}
In conclusion, we found that people who oppose hosting more refugees tend to overestimate the real numbers whereas people who favor hosting more refugees tend to underestimate the real numbers. However, our data do not support a correlation between precision of estimate and attitude. We cannot tell from our data if this means that subtle perception rather than knowledge influences people's attitude towards refugees. We therefore recommend further investigating this question with a more differentiated questionnaire that incorporates a well-defined proxy for the interviewees knowledge on refugees.\\
While gender and age did not significantly influence attitude in our study, answers differed significantly among places where the interviews were conducted. When reproducing this study, places for the interviews should be selected so that they represent an average of the population. Furthermore, we recommend using printed questionnaires to ensure objective and honest answers.


It is widely assumed that spatial structure allows for the evolution of cooperation in PD games. However, this goes only for a small range of cost-benefit ratios. Cooperation is only evolutionarily stable for r-values $ \leq 0.09$, for bigger r-values it disappears from the population.\\

In spatial HD games: small neighborhoods = bigger profit when r is small, and small neighborhoods = bigger disadvantage when r is big.

In spatial PD games: 

Found anomality

r 0.03, nb4 -> stable at propC 0.35 after 2500 steps -> stable
r 0.065, nb4 -> cooperators die after 900 steps -> unstable
r 0.03, nb8 -> stable at propC 0.77 after 10000 steps -> stable
r 0.065, nb8 -> stable at propC 0.55 after 2500 steps -> stable
r 0.03, nb12 -> stable at propC 0.825 after 7500 steps -> stable
r 0.065, nb12 -> randomly oscillating around propC 0.33 after 10000 steps -> half-stable
r 0.03, nb24 -> stable at propC 0.85 after 4000 steps -> stable
r 0.065, nb 24 -> cooperators die after 7400 steps -> unstable


fixed r: at 0.03 and 0.065
varying neighborhood size
5000 steps, 5 repetitions

\end{comment}





\section*{Acknowledgements}
First of all we want to thank all participants of the interviews who have made the whole study possible. We also want to thank the R core team for their great statistics software. Sincere thanks are due to our teachers Dr. Florian Hartig and Paul Bauche for their competent support and advice. Special thanks go to Lisa Rubin for her helpful review of the manuscript.



\onecolumn

\newpage
\newpage

\bibliographystyle{chicago}
\bibliography{EGT_refs}


\end{document}